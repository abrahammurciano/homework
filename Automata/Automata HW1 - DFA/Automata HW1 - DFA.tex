\documentclass[fleqn]{article}
\usepackage[margin=3cm]{geometry}   % shrink margins
\usepackage[margin=3cm]{caption}   % shrink captions
\usepackage{amsmath}    % math equation environments
\usepackage{amssymb}    % math symbols such as natural numbers N.

\newcommand{\N}{\mathbb{N}}
\newcommand{\lang}{\mathcal{L}}

\newenvironment{answers}{ % same as enumerate but with more space between each answer
	\begin{enumerate}
		\setlength{\itemsep}{\bigskipamount}
}{\end{enumerate}}

\newcommand\Item[1][]{ % custom \Item command for block math
  \ifx\relax#1\relax  \item \else \item[#1] \fi
  \abovedisplayskip=0pt\abovedisplayshortskip=0pt~\vspace*{-\baselineskip}}

%\usepackage{multicol}	% can be used to put enumerate in columns. Usage: \begin{multicols}{NumCols}\begin{enumerate}...

\usepackage{tikz}	% for diagrams
\usetikzlibrary{positioning}
\usetikzlibrary{arrows}
\usetikzlibrary{shapes}
\usetikzlibrary{quotes}

% same as tikzpicture but for state diagrams
% Usage:
%	the start state must have the style "start" (so the start arrow can point to it)
%	the start state cannot have an id, it will be given the id "start"
%	there can only be one start state
%		e.g. \node[state, start] {node content};
%	all state nodes should have style "state"
%		e.g. \node[state] (my_node_id) {node content};
%	all accepted state nodes should also have style "accepted"
%		e.g. \node[state, accepted] (my_node_id) {node content};
%	all input arrows should have style "input"
%		and optionally "bend left=40", "bend right=90", "loop", "loop right", etc. for curved arrows
%	all input arrows should be labelled with the inputs that lead that way
%		e.g. \draw[input, bend_left] (source) to["{possible inputs}" above,sloped] (destination);
%	an arror starte should be given the style "error" and it will be given the id "error"
\newenvironment{statediagram}{
	\begin{tikzpicture}[
		state/.style={circle, draw=blue!60, fill=blue!5, very thick, node distance=1.5cm},
		start/.style={name=start},
		accepted/.style={double, double distance=0.5mm},
		error/.style={draw=red!60, fill=red!5, name=error},
		input/.style={->, thick, shorten >= 1mm, shorten <= 1mm},
	]
}{
	\draw[input] (start.west)+(-1cm,0) to (start.west);
	\end{tikzpicture}
}

\newenvironment{statediagramfigure}{
	\begin{figure}[h]
		\centering
		\begin{statediagram}
}{\end{statediagram} \end{figure}}

% Math mode in tables. Usage: use column type C
% \usepackage{array}   % for \newcolumntype macro
% \newcolumntype{C}{>{$}c<{$}} % math-mode version of "c" column type

% paragraph indentation within enumerations
\usepackage{enumitem}
\setlist{parsep=4pt,listparindent=\parindent}

\newcommand{\prefix}{\sqsubseteq}
\newcommand{\suffix}{\sqsupseteq}

\title{Automata \& Formal Languages \\
\medskip
\large Homework 1 - Deterministic Finite Automata}
\author{Abraham Murciano}

\begin{document}

\maketitle

\begin{answers}

	\Item % 1
	\[\lang = \{w \in \{\text{a}, \text{b}, \text{c}\}^* : (\exists n \in \N : |W| = 2n)\}\]
	\begin{statediagramfigure}
		\node[state, accepted, start] {\(q_0\)};
		\node[state, right=of start] (odd) {\(q_1\)};
		\draw[input, bend left=40] (start) to["{a,b,c}" below] (odd);
		\draw[input, bend left=40] (odd) to["{a,b,c}" above] (start);
	\end{statediagramfigure}

	\Item % 2
	\[\lang = \{w \in \{\text{a}, \text{b}, \text{c}\}^* : \text{abc} \prefix w\}\]
	\begin{statediagramfigure}
		\node[state, start] {\(q_0\)};
		\node[state, right=of start] (a) {\(q_1\)};
		\node[state, right=of a] (ab) {\(q_2\)};
		\node[state, accepted, right=of ab] (abc) {\(q_3\)};
		\node[state, error, below=1cm of start] {\(q_4\)};

		\draw[input] (start) to["{a}" above] (a);
		\draw[input] (a) to["{b}" above] (ab);
		\draw[input] (ab) to["{c}" above] (abc);
		\draw[input, loop right] (abc) to["{a,b,c}" right] (abc);
		\draw[input] (start) to["{b, c}" left] (error);
		\draw[input, bend left=20] (a) to["{a, c}" above,sloped] (error);
		\draw[input, bend left=40] (ab) to["{a, b}" below,sloped] (error);
		\draw[input, loop left] (error) to["{a,b,c}" left] (error);
	\end{statediagramfigure}

	\Item % 3
	\[\lang = \{w \in \{\text{a}, \text{b}, \text{c}\}^* : \text{abc} \suffix w\}\]
	\begin{statediagramfigure}
		\node[start, state] {\(q_0\)};
		\node[state, right=of start] (a) {\(q_1\)};
		\node[state, right=of a] (ab) {\(q_2\)};
		\node[state, accepted, right=of ab] (abc) {\(q_3\)};

		\draw[input] (start) to["{a}" above] (a);
		\draw[input, loop below] (start) to["{b,c}" below] (start);

		\draw[input] (a) to["{b}" above] (ab);
		\draw[input, loop below] (a) to["{a}" below] (a);
		\draw[input, bend left=40] (a) to["{c}" below] (start);

		\draw[input] (ab) to["{c}" above] (abc);
		\draw[input, bend left=40] (ab) to["{a}" below] (a);
		\draw[input, bend left=60] (ab) to["{b}" below] (start);

		\draw[input, bend right=30] (abc) to["{a}" above] (a);
		\draw[input, bend right=40] (abc) to["{b,c}" below] (start);
	\end{statediagramfigure}

	For this language, the word ``aaaabc'' belongs in the language, but ``aaa'' does not.

	\pagebreak
	\Item % 4
	\[\lang = \{w \in \{\text{a}, \text{b}, \text{c}\}^* : \text{aaa} \not\prefix w\}\]
	\begin{statediagramfigure}
		\node[start, state] {\(q_0\)};
		\node[state, right=of start] (a) {\(q_1\)};
		\node[state, right=of a] (aa) {\(q_2\)};
		\node[state, right=of aa] (aaa) {\(q_3\)};
		\node[state, accepted, below=of a] (ok) {\(q_4\)};

		\draw[input] (start) to["{a}" above] (a);
		\draw[input, bend right=30] (start) to["{b,c}" below,sloped] (ok);

		\draw[input] (a) to["{a}" above] (aa);
		\draw[input] (a) to["{b,c}" above,sloped] (ok);

		\draw[input] (aa) to["{a}" above] (aaa);
		\draw[input, bend left=30] (aa) to["{b,c}" below,sloped] (ok);

		\draw[input, loop right] (aaa) to["{a,b,c}" right] (aaa);

		\draw[input, loop below] (ok) to["{a,b,c}" below] (ok);
	\end{statediagramfigure}

	\Item % 5
	\[\lang = \{w \in \{\text{a}, \text{b}, \text{c}\}^* : (\exists u,v \in \{\text{a}, \text{b}, \text{c}\}^* : w = u \text{ab} v \lor w = u \text{ba} v)\}\]
	\begin{statediagramfigure}
		\node[state, start] {\(q_0\)};
		\node[state, above right=of start] (a) {\(q_1\)};
		\node[state, below right=of start] (b) {\(q_2\)};
		\node[state, accepted, right=of a] (ab) {\(q_3\)};
		\node[state, accepted, right=of b] (ba) {\(q_4\)};

		\draw[input] (start) to["{a}" below,sloped] (a);
		\draw[input] (start) to["{b}" above,sloped] (b);
		\draw[input, loop above] (start) to["{c}" above] (start);

		\draw[input] (a) to["{b}"] (ab);
		\draw[input, loop below] (a) to["{a}" below] (a);
		\draw[input, bend right=30] (a) to["{c}" above,sloped] (start);

		\draw[input] (b) to["{a}"] (ba);
		\draw[input, loop above] (b) to["{b}"] (b);
		\draw[input, bend left=30] (b) to["{c}" below,sloped] (start);

		\draw[input, loop right] (ab) to["{a,b,c}"] (ab);
		\draw[input, loop right] (ba) to["{a,b,c}"] (ba);
	\end{statediagramfigure}

	\Item % 6
	\[\lang = \{\text{abba}\} \text{ where } \Sigma = \{\text{a}, \text{b}, \text{c}\}\]
	\begin{statediagramfigure}
		\node[state, start] {\(q_0\)};
		\node[state, right=of start] (a) {\(q_1\)};
		\node[state, right=of a] (ab) {\(q_2\)};
		\node[state, right=of ab] (abb) {\(q_3\)};
		\node[state, accepted, right=of abb] (abba) {\(q_4\)};
		\node[state, error, below=of ab] {\(q_5\)};

		\draw[input] (start) to["{a}"] (a);
		\draw[input, bend right=20] (start) to["{b,c}" below,sloped] (error);

		\draw[input] (a) to["{b}"] (ab);
		\draw[input, bend right=10] (a) to["{a,c}" below,sloped] (error);

		\draw[input] (ab) to["{b}"] (abb);
		\draw[input] (ab) to["{a,c}"] (error);

		\draw[input] (abb) to["{a}"] (abba);
		\draw[input, bend left=10] (abb) to["{b,c}" below,sloped] (error);

		\draw[input, bend left=20] (abba) to["{a,b,c}" below,sloped] (error);

		\draw[input, loop below] (error) to["{a,b,c}" below] (error);
	\end{statediagramfigure}

\end{answers}

\end{document}
