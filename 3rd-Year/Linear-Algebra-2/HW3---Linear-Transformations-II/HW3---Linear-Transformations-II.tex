\documentclass{article}
\usepackage{amsmath}    % math equation environments
\usepackage{amssymb}    % math symbols such as natural numbers N.
\usepackage{graphicx}

\newenvironment{answers}{ % same as enumerate but with more space between each answer
	\begin{enumerate}
		\setlength{\itemsep}{\bigskipamount}
}{\end{enumerate}}

% paragraph indentation within enumerations
\usepackage{enumitem}
\setlist{parsep=4pt,listparindent=\parindent}

\newcommand{\R}{\mathbb{R}}
\newcommand{\C}{\mathbb{C}}
\newcommand{\st}{\text{ such that }}
\newcommand{\img}{\operatorname{Im}}

\title{Linear Algebra 2 \\
\medskip
\large Homework 3 -- Linear Transformations II}
\author{Abraham Murciano}

\begin{document}

\maketitle

\begin{answers}

	\item
		For the following transformations, our task is to find their matrix representation using standard bases, if the transformation is linear.
		\begin{enumerate}
			\item[(b)]
				\(T : \R^2 \to \R^2 \st T(x, y) = (2x + |y|, -2y)\)

				This transformation is not linear.

			\item[(d)]
				\(T : \R^3 \to \R \st T(x, y, z) = y\)

				Since \(T\) maps from \(\R^3\) to \(\R\), the matrix must have dimensions such that when mutiplying it by a vector in \(\R^3\), the result must be in \(\R\). Therefore its dimensions must be three by one. And since we are only interested in the result containing \(y\), we can use the following matrix which multiplies any input vector \((x, y, z)\) as follows.
				\begin{equation*}
					\left[ \begin{matrix}
							0 & 1 & 0
						\end{matrix} \right]
					\times
					\left[ \begin{matrix}
							x \\ y \\ z
						\end{matrix} \right]
					=
					\left[ \begin{matrix}
							y
						\end{matrix} \right]
				\end{equation*}

			\item[(g)]
				\(T : \C^2 \to \C^2 \st T(z_1, z_2) = (|z_1 + z_2|, z_1 - 2z_2)\) where \(\C^2\) is a vector space over \(\C\).

				This is not a linear transformation either.
		\end{enumerate}

	\item[2.]
		Let \(T : \R^3 \to \R^2\) be the transformation defined by \(T(x, y, z) = (x - 2y, 2x + y - z)\).
		\begin{enumerate}
			\item
				To show that this transformation is linear, we must show that it is commutative with addition and with scalar mutiplication.

				\begin{align*}
					T(\vec{u} + \vec{v}) & = ((x_u + x_v) - 2(y_u + y_v), \\
					                     & 2(x_u + x_v) + (y_u + y_v) - (z_u + z_v)) \\
					                     & = (x_u + x_v - 2y_u - 2y_v, 2x_u + 2x_v + y_u + y_v - z_u - z_v) \\
					                     & = (x_u - 2y_u, 2x_u + y_u - z_u) + (x_v - 2y_v, 2x_v + y_v - z_v) \\
					                     & = T(\vec{u}) + T(\vec{v}) \\
				\end{align*}
				\begin{align*}
					T(c\vec{u}) & = (cx - 2cy, 2cx + cy - cz) \\
					            & = (c(x - 2y), c(2x + y - z)) \\
					            & = c(x - 2y, 2x + y - z) \\
					            & = cT(\vec{u})
				\end{align*}

			\item
				The inverse image of \((1, 3)\) under \(T\), i.e. \(\{v \in \R^3 : T(v) = (1,3)\}\), can be described with the following system of equations.
				\begin{gather*}
					x - 2y = 1 \\
					2x + y - z = 3
				\end{gather*}
				Solving this system gives us the line \((x, y, z) = (\frac{7}{5},\frac{1}{5},0) + t(-\frac{2}{5},\frac{1}{5},\frac{1}{5})\).

			\item
				To find a basis for the kernel of \(T\), it suffices to find one vector \(\vec{v} \in \ker T\), since we know from the previous question that the dimension of the kernel must be one, i.e. a line.

				Letting \(x := 2\) and \(y := 1\), we have \(x - 2y = 0\). Now to obtain \(2x + y -z = 0\) as well, we can simply set \(z := 2x+y = 5\). Therefore \((2, 1, 5)\) is a basis of the kernel of \(T\).

			\item
				Since the dimension of the kernel is one, and the domain of the transformation has a dimension of three, we must conclude that the image of \(T\) has a dimension of \(\dim(\R^3) - \dim(\ker T) = 3 - 1 = 2\). Therefore any two independent vectors in the image of \(T\) will form a basis for the image.

				To find two such vectors, we can apply \(T\) to two vectors in the domain, then verify that they are indeed independent. We shall choose the vectors \((1, 0, 0)\) and \((0, 0, 1)\). Applying \(T\) to these gives us a basis \(\{(1, 2), (0, -1)\}\), which is indeed independent.
		\end{enumerate}

	\item
		Consider the transformation \(T:\R^4 \to \R^2\) whose matrix is the following.
		\begin{equation*}
			[T] = \left[ \begin{matrix}
					1  & 1 & 3 & 4 \\
					-2 & 2 & 1 & 1
				\end{matrix} \right]
		\end{equation*}
		\begin{enumerate}
			\item
				\(T\) cannot be one to one, since we shall show in part (c) that it is onto. And if it would be both one to one and onto, then it would be invertible, which is impossible since the range and domain have different dimensions.
			\item
			\item
				\(T\) must be onto, since at two of the vectors in its image, for example \((1, -2)\) and \((1, 2)\), span the entire range \(\R^2\), since they are linearly independent.
		\end{enumerate}
\end{answers}

\end{document}