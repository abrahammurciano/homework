\documentclass{article}
\usepackage{amsmath}    % math equation environments
\usepackage{amssymb}    % math symbols such as natural numbers N.
\usepackage{graphicx}

\newenvironment{answers}{ % same as enumerate but with more space between each answer
	\begin{enumerate}
		\setlength{\itemsep}{\bigskipamount}
}{\end{enumerate}}

% paragraph indentation within enumerations
\usepackage{enumitem}
\setlist{parsep=4pt,listparindent=\parindent}

\newcommand{\R}{\mathbb{R}}
\newcommand{\C}{\mathbb{C}}
\newcommand{\st}{\text{ such that }}
\newcommand{\img}{\operatorname{Im}}
\newcommand{\tr}{\operatorname{tr}}

\title{Linear Algebra 2 \\
\medskip
\large Homework 3 -- Linear Transformations II}
\author{Abraham Murciano}

\begin{document}

\maketitle

\begin{answers}

	\item
		For the following transformations, our task is to find their matrix representation using standard bases, if the transformation is linear.
		\begin{enumerate}
			\item[(b)]
				\(T : \R^2 \to \R^2 \st T(x, y) = (2x + |y|, -2y)\)

				This transformation is not linear.

			\item[(d)]
				\(T : \R^3 \to \R \st T(x, y, z) = y\)

				Since \(T\) maps from \(\R^3\) to \(\R\), the matrix must have dimensions such that when mutiplying it by a vector in \(\R^3\), the result must be in \(\R\). Therefore its dimensions must be three by one. And since we are only interested in the result containing \(y\), we can use the following matrix which multiplies any input vector \((x, y, z)\) as follows.
				\begin{equation*}
					\left[ \begin{matrix}
							0 & 1 & 0
						\end{matrix} \right]
					\times
					\left[ \begin{matrix}
							x \\ y \\ z
						\end{matrix} \right]
					=
					\left[ \begin{matrix}
							y
						\end{matrix} \right]
				\end{equation*}

			\item[(g)]
				\(T : \C^2 \to \C^2 \st T(z_1, z_2) = (|z_1 + z_2|, z_1 - 2z_2)\) where \(\C^2\) is a vector space over \(\C\).

				This is not a linear transformation either.
		\end{enumerate}

	\item[2.]
		Let \(T : \R^3 \to \R^2\) be the transformation defined by \(T(x, y, z) = (x - 2y, 2x + y - z)\).
		\begin{enumerate}
			\item
				To show that this transformation is linear, we must show that it is commutative with addition and with scalar mutiplication.

				\begin{align*}
					T(\vec{u} + \vec{v}) & = ((x_u + x_v) - 2(y_u + y_v), \\
					                     & 2(x_u + x_v) + (y_u + y_v) - (z_u + z_v)) \\
					                     & = (x_u + x_v - 2y_u - 2y_v, 2x_u + 2x_v + y_u + y_v - z_u - z_v) \\
					                     & = (x_u - 2y_u, 2x_u + y_u - z_u) + (x_v - 2y_v, 2x_v + y_v - z_v) \\
					                     & = T(\vec{u}) + T(\vec{v}) \\
				\end{align*}
				\begin{align*}
					T(c\vec{u}) & = (cx - 2cy, 2cx + cy - cz) \\
					            & = (c(x - 2y), c(2x + y - z)) \\
					            & = c(x - 2y, 2x + y - z) \\
					            & = cT(\vec{u})
				\end{align*}

			\item
				The inverse image of \((1, 3)\) under \(T\), i.e. \(\{v \in \R^3 : T(v) = (1,3)\}\), can be described with the following system of equations.
				\begin{gather*}
					x - 2y = 1 \\
					2x + y - z = 3
				\end{gather*}
				Solving this system gives us the line \((x, y, z) = (\frac{7}{5},\frac{1}{5},0) + t(-\frac{2}{5},\frac{1}{5},\frac{1}{5})\).

			\item
				To find a basis for the kernel of \(T\), it suffices to find one vector \(\vec{v} \in \ker T\), since we know from the previous question that the dimension of the kernel must be one, i.e. a line.

				Letting \(x := 2\) and \(y := 1\), we have \(x - 2y = 0\). Now to obtain \(2x + y -z = 0\) as well, we can simply set \(z := 2x+y = 5\). Therefore \((2, 1, 5)\) is a basis of the kernel of \(T\).

			\item
				Since the dimension of the kernel is one, and the domain of the transformation has a dimension of three, we must conclude that the image of \(T\) has a dimension of \(\dim(\R^3) - \dim(\ker T) = 3 - 1 = 2\). Therefore any two independent vectors in the image of \(T\) will form a basis for the image.

				To find two such vectors, we can apply \(T\) to two vectors in the domain, then verify that they are indeed independent. We shall choose the vectors \((1, 0, 0)\) and \((0, 0, 1)\). Applying \(T\) to these gives us a basis \(\{(1, 2), (0, -1)\}\), which is indeed independent.
		\end{enumerate}

	\item[4.]
		Consider the transformation \(T:\R^4 \to \R^2\) whose matrix is the following.
		\begin{equation*}
			[T] = \left[ \begin{matrix}
					1  & 1 & 3 & 4 \\
					-2 & 2 & 1 & 1
				\end{matrix} \right]
		\end{equation*}
		\begin{enumerate}
			\item
				\(T\) cannot be one to one, since we shall show in part (c) that it is onto. And if it would be both one to one and onto, then it would be invertible, which is impossible since the range and domain have different dimensions.

			\item
				Since the transformation is onto, the dimension of the image is two. Therefore the dimension of the kernel is \(4-2=2\). So it suffices to find two independent vectors in \(\ker T\), and those will form a basis.

				To find these, we can treat the matrix as a homogeneous one, to indicate that \(T(x_1, x_2, x_3, x_4) = (0, 0)\), then row reduce it to reduced row echelon form.
				\begin{equation*}
					\left[ \begin{matrix}
							1 & 0 & \frac{5}{4} & \frac{7}{4} \\[6pt]
							0 & 1 & \frac{7}{4} & \frac{9}{4}
						\end{matrix} \right]
				\end{equation*}

				This gives us the following system of equations which holds when \([T]\vec{x} = (0,0)\).
				\begin{gather*}
					x = -\frac{5}{4}x_3 - \frac{7}{4}x_4 \\
					y = -\frac{7}{4}x_3 - \frac{9}{4}x_4
				\end{gather*}

				In matrix form, this equation can be written like this.
				\begin{equation*}
					\vec{x} = \left[\begin{matrix}
							x_1 \\ x_2 \\ x_3 \\ x_4
						\end{matrix} \right]
					= \left[ \begin{matrix}
							-\frac{5}{4}x_3 - \frac{7}{4}x_4 \\[6pt] -\frac{7}{4}x_3 - \frac{9}{4}x_4 \\ x_3 \\ x_4
						\end{matrix} \right]
					= \left[ \begin{matrix}
							-\frac{5}{4} \\[6pt] -\frac{7}{4} \\ 1 \\ 0
						\end{matrix} \right]x_3 + \left[ \begin{matrix}
							- \frac{7}{4} \\[6pt] - \frac{9}{4} \\ 0 \\ 1 \end{matrix} \right]x_4
				\end{equation*}

				This means that for all \(x_3, x_4 \in \R\), we can find some vector \(\vec{x} \in \ker T\) according to the previous equation. Therefore if we choose \(x_3 = 1, x_4 = 0\) we get one vector in the kernel, \((-\frac{5}{4}, -\frac{7}{4}, 1, 0)\). And choosing \(x_3 = 0, x_4 = 1\) gives us another vector in the kernel, \((- \frac{7}{4}, - \frac{9}{4}, 0, 1)\). Since these are independent, they form a basis for the kernel.

			\item
				\(T\) must be onto, since two of the vectors in its image, for example \((1, -2)\) and \((1, 2)\), span the entire range \(\R^2\), since they are linearly independent.
		\end{enumerate}

	\item[5.]
		\begin{enumerate}
			\item[(b)]
				Given the transformation \(T : \C^2 \to \C^2\) over the vector space \(\C\) represented by the matrix below, we are tasked with proving that it is invertible, and we seek the inverse transformation matrix.
				\begin{equation*}
					[T] = \left[ \begin{matrix}
							i & 2i \\
							2 & i
						\end{matrix} \right]
				\end{equation*}

				A linear transformation is invertible if and only if its transformation matrix is invertible. And a transformation matrix is invertible if and only if its determinant is not zero.
				\begin{equation*}
					\det [T] = i \times i - 2 \times 2i = i^2 - 4i = -1 - 4i \neq 0
				\end{equation*}
				\begin{equation*}
					\left[ T^{-1} \right] = [T]^{-1}
					= \frac{1}{-1-4i}\left[ \begin{matrix} i & -2i \\ -2 & i \end{matrix} \right]
					= \frac{4i-1}{17}\left[ \begin{matrix} i & -2i \\ -2 & i \end{matrix} \right]
				\end{equation*}
		\end{enumerate}

	\item[8.]
		Let \(V\) be the set of all matrices of the following form, where \(a,b \in \R\).
		\begin{equation*}
			\left[ \begin{matrix}
					a & b \\ -b & a
				\end{matrix} \right]
		\end{equation*}
		We shall take for granted that \(V\) is a vector space over \(\R\). Now consider the following transformation.
		\begin{equation*}
			T : \C \to V \st T(a + bi) = \left[ \begin{matrix} a & b \\ -b & a \end{matrix} \right]
		\end{equation*}
		\begin{enumerate}
			\item
				To prove that \(T\) is a linear transformation, we must show that the following two equations hold.
				\begin{gather*}
					cT(a+bi) = T(c(a+bi)) \\
					T(a_1 + b_1i) + T(a_2 + b_2i) = T(a_1 + b_1i + a_2 + b_2i)
				\end{gather*}
				\begin{equation*}
					cT(a+bi) = c \left[ \begin{matrix} a & b \\ -b & a \end{matrix} \right] = \left[ \begin{matrix} ca & cb \\ -cb & ca \end{matrix} \right] = T(ca + cbi) = T(c(a+bi))
				\end{equation*}
				\begin{align*}
					T(a_1 + b_1i) + T(a_2 + b_2i) & = \left[ \begin{matrix} a_1 & b_1 \\ -b_1 & a_1 \end{matrix} \right] + \left[ \begin{matrix} a_2 & b_2 \\ -b_2 & a_2 \end{matrix} \right] \\
					                              & = \left[ \begin{matrix} a_1+a_2 & b_1+b_2 \\ -(b_1+b_2) & a_1+a_2 \end{matrix} \right] \\
					                              & = T((a_1 + a_2) + (b_1 + b_2)i) \\
					                              & = T(a_1 + b_1i + a_2 + b_2i)
				\end{align*}

			\item
				We must now prove that \(T\) is multplicative, meaning that we have \(\forall z,w \in \C, T(zw) = T(z)T(w)\).
				\begin{align*}
					z        & := a_1 + b_1i \\
					w        & := a_2 + b_2i \\
					T(z)T(w) & = \left[ \begin{matrix} a_1 & b_1 \\ -b_1 & a_1 \end{matrix} \right]\left[ \begin{matrix} a_2 & b_2 \\ -b_2 & a_2 \end{matrix} \right] \\
					         & = \left[ \begin{matrix} a_1a_2-b_1b_2 & a_1b_2+b_1a_2 \\ -(a_1b_2+b_1a_2) & a_1a_2-b_1b_2 \end{matrix} \right] \\
					         & = T((a_1a_2-b_1b_2) + (a_1b_2+b_1a_2)i) \\
					         & = T(a_1a_2 - b_1b_2 + a_1b_2i + b_1a_2i) \\
					         & = T(a_1(a_2 + b_2i) + b_1(-b_2 + a_2i)) \\
					         & = T(a_1w + b_1(a_2i - b_2)) \\
					         & = T(a_1w + b_1i(a_2 - \frac{b_2}{i})) \\
					         & = T(a_1w + b_1i(a_2 + b_2i)) \\
					         & = T(a_1w + b_1iw) \\
					         & = T((a_1 + b_1i)w) \\
					         & = T(zw)
				\end{align*}
		\end{enumerate}

	\item[10.]
		\begin{enumerate}
			\item[(b)]
				There does not exist a unique linear transformation \(T : \R^2 \to \R^2\) such that \(T(1, 2) = (1, 1)\), \(T(1, -1) = (2, -1)\), and \(T(1, -7) = (8, -5)\).

				To prove this, let \([T]\) be defined as follows.
				\begin{equation*}
					[T] := \left[ \begin{matrix}
							a & b \\ c & d
						\end{matrix} \right]
				\end{equation*}
				With the above constraints we can form the following equations by multiplying the above matrix by the given input vectors and setting the result to the given output vectors.
				\begin{align*}
					a + 2b & = 1  &
					c + 2d & = 1 \\
					a - b  & = 2  &
					c - d  & = -1 \\
					a - 7b & = 8  &
					c - 7d & = -5
				\end{align*}
				Using the first two equations on the left, we can conclude that \(a = \frac{5}{3}\) and \(b = -\frac{1}{3}\). Similarly, using the first two equations on the right we obtain \(c = -\frac{1}{3}\) and \(d = \frac{2}{3}\).

				However, the last equation on the left does not fit with the calculated values for \(a\) and \(b\), therefore we must conclude that there exists no such linear transformation \(T\).

			\item[(d)]
				There does not exist a unique linear transformation \(T : M_2(\R) \to \R^3\) such that
				\begin{equation*}
					T\left( \left[ \begin{matrix} 1 & 0 \\ 0 & 0 \end{matrix} \right] \right) = (1, 2, 3)\quad \land\quad T\left( \left[ \begin{matrix} 0 & 1 \\ 0 & 0 \end{matrix} \right] \right) = (1, 2, 3)
				\end{equation*}

				These two constraints cannot be enough to uniquely define \(T\), since they are only defined on two elements in \(M_2(\R)\).Yet in order to uniquely define \(T\), we must define it on a basis of the domain, which has a dimension of four. Therefore two elements are not enough, so \(T\) is not unique.

			\item[(h)]
				There exists a unique linear transformation \(T : \R^3 \to \R^3\) such that the image of the plane \(P = \{(2,2,1) + s(1,1,1) + t(1,4,0)\}\) is the point \((0,3,1)\).

				For \(T\) to meet these constraints, the following must hold true.
				\begin{equation*}
					\forall s,t \in \R,\ T((2,2,1) + s(1,1,1) + t(1,4,0)) = (0,3,1)
				\end{equation*}
				By linearity of \(T\), we can then say the following.
				\begin{equation*}
					\forall s,t \in \R,\ T(2,2,1) + s + tT(1,4,0) = (0,3,1)
				\end{equation*}
				Yet this can only be true if \(T(1,1,1) = T(1,4,0) = (0,0,0)\) and \(T(2,2,1) = (0,3,1)\).

				Thus we have defined \(T\) on three vectors in \(\R^3\), and these form a basis over \(\R^3\), since there are as many of them as the dimension and they are independent. Hence we conclude that \(T\) is indeed unique.
		\end{enumerate}

	\item[11.]
		The trace of a square matrix \(A\), denoted by \(\tr(A)\), is defined by the sum of the elements in the main diagonal of \(A\). Let us define the transformation \(T : M_3(\R) \to \R\) by \(T(A) = tr(A)\).
		\begin{enumerate}
			\item \(T\) is a linear transformation.
			\item \(T\) is onto.
			\item \(T\) is not one to one.
			\item \(\dim \ker T = 2\).
		\end{enumerate}

\end{answers}

\end{document}