\documentclass{article}
\usepackage{amsmath}    % math equation environments
\usepackage{amssymb}    % math symbols such as natural numbers N.

\newenvironment{answers}{ % same as enumerate but with more space between each answer
	\begin{enumerate}
		\setlength{\itemsep}{\bigskipamount}
}{\end{enumerate}}

% paragraph indentation within enumerations
\usepackage{enumitem}
\setlist{parsep=4pt,listparindent=\parindent}

\newcommand{\R}{\mathbb{R}}
\newcommand{\C}{\mathbb{C}}
\newcommand{\st}{\text{ such that }}

\title{Linear Algebra 2 \\
\medskip
\large Homework 4 -- Coordinates and Change of Basis}
\author{Abraham Murciano}

\begin{document}

\maketitle

Throughout this document, \(E = (e_1, e_2, \dots, e_n)\) shall denote the standard basis of \(\R^n\) or \(\C^n\), as appropriate according to the context in which it is used.

\begin{answers}

	\item
	We are given the transformation \(T : \R^2 \to \R^2 \st T(x,y) = (2x-y, x-3y)\).

	\begin{enumerate}
		\item
		      To find \([T]^E_E\), we simply construct a matrix whose column vectors are the transformation of each of the elements in the standard basis.
		      \begin{equation*}
			      \left[ \begin{matrix}
					      \left( \begin{matrix}
						      \\ T(e_1) \\ {}
					      \end{matrix} \right) &
					      \left( \begin{matrix}
						      \\ T(e_2) \\ {}
					      \end{matrix} \right)
				      \end{matrix} \right]
			      = \left[ \begin{matrix}
					      2 & -1 \\
					      1 & -3
				      \end{matrix} \right]
		      \end{equation*}
		\item
		      To find \([T]^B_C\), where \(B = ((1,1), (2,5))\) and \(C=((2,2), (0,1))\),
	\end{enumerate}

\end{answers}

\end{document}