\documentclass{article}
\usepackage{amsmath}    % math equation environments
\usepackage{amssymb}    % math symbols such as natural numbers N.

\newenvironment{answers}{ % same as enumerate but with more space between each answer
	\begin{enumerate}
		\setlength{\itemsep}{\bigskipamount}
}{\end{enumerate}}

% paragraph indentation within enumerations
\usepackage{enumitem}
\setlist{parsep=4pt,listparindent=\parindent}

\newcommand{\R}{\mathbb{R}}
\newcommand{\C}{\mathbb{C}}
\newcommand{\st}{\text{ such that }}

\title{Linear Algebra 2 \\
\medskip
\large Homework 4 -- Coordinates and Change of Basis}
\author{Abraham Murciano}

\begin{document}

\maketitle

Throughout this document, \(E = (e_1, e_2, \dots, e_n)\) shall denote the standard basis of \(\R^n\) or \(\C^n\), as appropriate according to the context in which it is used.

\begin{answers}

	\item
		We are given the transformation \(T : \R^2 \to \R^2 \st T(x,y) = (2x-y, x-3y)\).

		\begin{enumerate}
			\item
				To find \([T]^E_E\), we simply construct a matrix whose column vectors are the transformation of each of the elements in the standard basis.
				\begin{equation*}
					\left[ \begin{matrix}
							\left( \begin{matrix}
								{} \\ T(e_1) \\ {}
							\end{matrix} \right) &
							\left( \begin{matrix}
								{} \\ T(e_2) \\ {}
							\end{matrix} \right)
						\end{matrix} \right]
					= \left[ \begin{matrix}
							2 & -1 \\
							1 & -3
						\end{matrix} \right]
				\end{equation*}
			\item
				To find \([T]^B_C\), where \(B = ((1,1), (2,5))\) and \(C=((2,2), (0,1))\), we construct a matrix whose column vectors are the transformation of each element of \(B\), denoted as coordinate vectors with respect to \(C\).
				\begin{gather*}
					\left[ \begin{matrix}
							\left( \begin{matrix}
								{} \\ [T(1,1)]_C \\ {}
							\end{matrix} \right) &
							\left( \begin{matrix}
								{} \\ [T(2,5)]_C \\ {}
							\end{matrix} \right)
						\end{matrix} \right] \\
					=\left[ \begin{matrix}
							\left( \begin{matrix}
								{} \\ [(1,-2)]_C \\ {}
							\end{matrix} \right) &
							\left( \begin{matrix}
								{} \\ [(-1,-13)]_C \\ {}
							\end{matrix} \right)
						\end{matrix} \right]
				\end{gather*}

				In order to represent the coordinates \((1, -2)\) and \((-1, -13)\) with respect to the basis \(C\), we may use the change of basis matrix to convert them from the standard basis to the basis \(C\).

				The change of basis matrix from the standard basis to any other basis is the inverse of a matrix whose column vectors are the elements of the basis we are converting to.

				Therefore to convert from the standard basis to the desired basis we can either multiply the inverse of this matrix by the coordinates to convert, or we can solve the following system of equations.
				\begin{gather*}
					\left[ \begin{matrix}
							2 & 0 \\ 2 & 1
						\end{matrix} \right]
					\left[ \begin{matrix}
							a \\ b
						\end{matrix} \right]
					= \left[ \begin{matrix}
							1 \\ 2
						\end{matrix} \right]
					\text{ or } \left[ \begin{matrix}
							-1 \\ -13
						\end{matrix} \right]
				\end{gather*}

				Forming a matrix out of this and subsequently row reducing it tells us that
				\begin{gather*}
					[(1,-2)]_C = \left( \frac{1}{2}, 1 \right) \\
					[(-1,-13)]_C = \left( -\frac{1}{2}, -14 \right)
				\end{gather*}

				Therefore the matrix \([T]^B_C\) can be written as follows.
				\begin{equation*}
					[T]^B_C = \left[ \begin{matrix}
							\frac{1}{2} & -\frac{1}{2} \\[3pt]
							1           & 14
						\end{matrix} \right]
				\end{equation*}
		\end{enumerate}

\end{answers}

\end{document}