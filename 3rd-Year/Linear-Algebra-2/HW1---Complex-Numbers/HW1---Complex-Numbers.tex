\documentclass[fleqn]{article}
\usepackage{amsmath}    % math equation environments
\usepackage{amssymb}    % math symbols such as natural numbers N.

\newenvironment{answers}{ % same as enumerate but with more space between each answer
	\begin{enumerate}
		\setlength{\itemsep}{\bigskipamount}
}{\end{enumerate}}

\newcommand\Item[1][]{ % custom \Item command for block math
  \ifx\relax#1\relax  \item \else \item[#1] \fi
  \abovedisplayskip=0pt\abovedisplayshortskip=0pt~\vspace*{-\baselineskip}}

% paragraph indentation within enumerations
\usepackage{enumitem}
\setlist{parsep=4pt,listparindent=\parindent}

\title{Linear Algebra 2 \\
\medskip
\large Homework 1 -- Complex Numbers}
\author{Abraham Murciano}

\begin{document}

\maketitle

\begin{answers}

	\item[2.]
		We are tasked with solving the following equations for \(z\).
		\begin{enumerate}
			\Item[(b)]
				\begin{align*}
					z^2                                            & = -10+20i \\
					(a + bi)^2                                     & = -10+20i \\
					a^2 + 2abi - b^2                               & = -10+20i \\
					a^2 -b^2                                       & = -10 \\
					2ab                                            & = 20 \\
					b                                              & = \frac{10}{a} \\
					a^2 -\left( \frac{10}{a} \right)^2             & = -10 \\
					a^2 - \frac{100}{a^2}                          & = -10 \\
					a^4 + 10a^2 - 100                              & = 0 \\
					\text{Let } t                                  & = a^2 \\
					t^2 + 10t - 100                                & = 0 \\
					t                                              & = -5 + 5 \sqrt{5} \\
					a = \pm\sqrt{-5+5\sqrt{5}}                     & = \pm\sqrt{5}\sqrt{\sqrt{5} - 1} \\
					b = \pm \frac{10}{\sqrt{5}\sqrt{\sqrt{5} - 1}} & = \pm \frac{2\sqrt{5}}{\sqrt{\sqrt{5} - 1}} \\
					z_1                                            & = \sqrt{5}\sqrt{\sqrt{5} - 1} + \frac{2\sqrt{5}}{\sqrt{\sqrt{5} - 1}}i \\
					z_2                                            & = -\sqrt{5}\sqrt{\sqrt{5} - 1} - \frac{2\sqrt{5}}{\sqrt{\sqrt{5} - 1}}i
				\end{align*}

			\Item[(d)]
				\begin{align*}
					z^2 + |z|^2                  & = 2 - 4i \\
					(a+bi)^2 + |a+bi|^2          & = 2 - 4i \\
					(a+bi)^2 + \sqrt{a^2+b^2}^2  & = 2 - 4i \\
					a^2 - b^2 + 2abi + a^2 + b^2 & = 2 - 4i \\
					2a^2 + 2abi                  & = 2 - 4i \\
					2a^2                         & = 2 \\
					2ab                          & = -4 \\
					a^2                          & = 1 \\
					b                            & = - \frac{2}{a} \\
					a                            & = \pm 1 \\
					b                            & = \mp 2 \\
					z_1                          & = 1 - 2i \\
					z_2                          & = -1 + 2i
				\end{align*}
		\end{enumerate}

	\item[3.]
		For the following properties of the complex conjugates and absolute values we must prove each property, using either Cartesian or polar form.
		\begin{enumerate}
			\Item[(a)]
				\begin{align*}
					\overline{z_1} + \overline{z_2} & = \overline{z_1 + z_2} \\
					\text{Let } z_1                 & = a_1 + b_1i \\
					\text{Let } z_2                 & = a_2 + b_2i \\
					\overline{z_1} + \overline{z_2} & = a_1 - b_1i + a_2 - b_2i \\
					                                & = (a_1 + a_2) - (b_1 + b_2)i \\
					                                & = \overline{(a_1 + a_2) + (b_1 + b_2)i} \\
					                                & = \overline{a_1 + b_1i + a_2 + b_2i} \\
					                                & =\overline{z_1 + z_2}
				\end{align*}

			\Item[(d)]
				\begin{align*}
					\overline{\overline{z}} & = z \\
					\text{Let } z           & = a + bi \\
					\overline{\overline{z}} & = \overline{\overline{a + bi}} \\
					                        & = \overline{a - bi} \\
					                        & = a + bi \\
					                        & = z
				\end{align*}
		\end{enumerate}

	\item[4.]
		We are asked to write each of the following complex numbers in their polar representation:
		\begin{enumerate}
			\Item[(b)]
				\begin{align*}
					1 - i  & = r (\cos \theta + i \sin \theta) \\
					r      & = |1 - i| = \sqrt{1^2 + (-1)^2} = \sqrt{2} \\
					\theta & = -\arctan(1) = -\frac{\pi}{4} \\
					1 - i  & = \sqrt{2}\left( \cos\left( -\frac{\pi}{4} \right) + i\sin\left( -\frac{\pi}{4} \right) \right) \\
					       & = \sqrt{2}\left( \cos\left( \frac{\pi}{4} \right) - i\sin\left( \frac{\pi}{4} \right) \right)
				\end{align*}

			\Item[(d)]
				\begin{equation*}
					4 - 4i = 4(1 - i) = 4\sqrt{2}\left( \cos\frac{\pi}{4} + i\sin\frac{\pi}{4} \right)
				\end{equation*}
		\end{enumerate}

	\item[5.]
		\begin{enumerate}
			\item[(c)]
				We are to express the following number in cartesian form.
				\begin{equation*}
					\left( \frac{1+2i}{-2+i} \right)^{2048} = \frac{1+2^{2048}}{2^{2048}+1} = 1
				\end{equation*}
		\end{enumerate}

	\item[6.]
		\begin{enumerate}
			\item[(b)]
				We seek to solve the following equation using the polar representation of complex numbers.
				\begin{align*}
					z^3 & = -2 + 2i = |-2+2i|e^{\frac{3\pi i}{4}} = 2\sqrt{2}e^{\frac{3\pi i}{4}} \\
					z_1 & = \left( 2\sqrt{2}e^{\frac{3\pi i}{4}} \right)^\frac{1}{3} = \sqrt{2}e^{\frac{\pi}{4}i}  = \sqrt{2}\left( \cos\frac{\pi}{4} + i\sin\frac{\pi}{4} \right) = 1 + i \\
					z_2 & = \left( 2\sqrt{2}e^{\left( \frac{3\pi}{4} + 2\pi \right) i} \right)^\frac{1}{3} = \sqrt{2}e^{\frac{11\pi i}{12}} \\
					    & =  \sqrt{2}\left( \cos\frac{11\pi}{12} + i\sin\frac{11\pi}{12} \right) = -\frac{1}{2} - \frac{\sqrt{3}}{2} + \left( \frac{\sqrt{3}}{2} - \frac{1}{2} \right) i \\
					z_3 & = \left( 2\sqrt{2}e^{\left( \frac{3\pi}{4} + 4\pi \right) i} \right)^\frac{1}{3} = \sqrt{2}e^{\frac{19\pi i}{12}} \\
					    & = \sqrt{2} \left( \cos\frac{19\pi}{12} + i\sin\frac{19\pi}{12} \right) = \frac{\sqrt{3}}{2} - \frac{1}{2} - \left( \frac{1}{2} + \frac{\sqrt{3}}{2} \right) i
				\end{align*}
		\end{enumerate}
\end{answers}

\end{document}