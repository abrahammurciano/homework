\documentclass{article}

\usepackage{amsmath}
\usepackage{amssymb}
\usepackage{algorithm}
\usepackage[noend]{algpseudocode}		% for algorithms in pseudo code. Usage: \begin{algorithmic}
\usepackage{slashbox}

\setlength{\parskip}{\smallskipamount}

\title{Analysis of Algorithms \\
\medskip
\large Homework 2}
\author{Abraham Murciano}

\begin{document}

\maketitle

\section{Matrix Multiplication}

We are given the following four matrices to be multiplied, as well as their sizes.

\begin{table}[h]
	\centering
	\begin{tabular}{|c|c|}
		\hline
		Matrix           & Size \\
		\hline
		\(\mathbf{A}_1\) & \(10 \times 30\) \\
		\(\mathbf{A}_2\) & \(30 \times 5\) \\
		\(\mathbf{A}_3\) & \(5 \times 60\) \\
		\(\mathbf{A}_4\) & \(60 \times 10\) \\
		\hline
	\end{tabular}
\end{table}

We are to apply the algorithm described in class in order to find the order in which to apply the associative matrix multiplications such that the number of scalar multiplications is minimal.

When multiplying two matrices, suppose these are of sizes \(r \times s\) and \(s \times t\), the resulting matrix would be of size \(r \times t\), meaning \(r \cdot t\) dot products must be calculated. And each of those dot products would be a sum of \(s\) scalar multiplications. Therefore the total number of scalar multiplications for these two matrices is \(r \cdot s \cdot t\).

When multiplying a chain of \(n\) matrixes, we can say that whatever the optimal order of performing the multiplications, if the last two matrices to be multiplied are
\begin{equation*}
	(\mathbf{A}_1 \times \dots \times \mathbf{A}_k) \times (\mathbf{A}_{k+1} \times \dots \times \mathbf{A}_n)
\end{equation*}
then the way to multiply \(\mathbf{A}_1 \times \dots \times \mathbf{A}_k\) must be optimal, and the same can be said about \(\mathbf{A}_{k+1} \times \dots \times \mathbf{A}_n\).

We denote by \(m_{i,j}\) the minimal number of scalar multiplications required to multiply matrices \(\mathbf{A}_i\) through \(\mathbf{A}_j\), where the dimensions of matrix \(\mathbf{A}_i\) is denoted by \(d_i \times d_{i+1}\). Now we can define \(m_{i,j}\).
\begin{equation*}
	m_{i,j} =
	\begin{cases}
		0                                                                                                      & i = j \\
		\displaystyle\min_{i \leq k < j} \left( m_{i,k} + m_{k+1, j} + d_i \cdot d_{k+1} \cdot d_{j+1} \right) & i < j
	\end{cases}
\end{equation*}

Let \(s_{i,j}\) be the value of \(k\) which gives the minimal number of multiplications in the definition of \(m_{i,j}\). Now we are ready to compute the initial question by calculating \(m_{1,4}\), \(s_{1,4}\), and all the intermediate results, as shown in table \ref{q1}.

\begin{table}[h]
	\centering
	\begin{tabular}{|c|cccc|}
		\hline
		\backslashbox{\(i\)}{\(j\)} & 1 & 2    & 3    & 4 \\
		\hline
		1                           & 0 & 1500 & 4500 & 5000 \\
		2                           & - & 0    & 9000 & 4500 \\
		3                           & - & -    & 0    & 3000 \\
		4                           & - & -    & -    & 0 \\
		\hline
	\end{tabular}
	\begin{tabular}{|c|cccc|}
		\hline
		\backslashbox{\(i\)}{\(j\)} & 1 & 2 & 3 & 4 \\
		\hline
		1                           & - & 1 & 2 & 2 \\
		2                           & - & - & 2 & 2 \\
		3                           & - & - & - & 3 \\
		4                           & - & - & - & - \\
		\hline
	\end{tabular}
	\caption{The computed final and intermediate results of \(m_{1,4}\) (left) and \(s_{1,4}\) (right)}
	\label{q1}
\end{table}

Thus we have shown that the optimal way to multiply \(\mathbf{A}_1 \times \mathbf{A}_2 \times \mathbf{A}_3 \times \mathbf{A}_4\) is \(((\mathbf{A}_1)(\mathbf{A}_2))((\mathbf{A}_3)(\mathbf{A}_4))\) with a total of 5000 scalar multiplications.

\section{Longest Non-Decreasing Subsequence}

\paragraph{The problem.} Given a sequence of natural numbers \(S = (s_1, s_2, \dots, s_n)\), we are to create an algorithm to find the longest monotonically non-decreasing subsequence as well as its length.

\newcommand\concat{\mathbin{+\mkern-10mu+}}
\paragraph{Some notation.} Let us denote by \(X'_z\) the sequence of the first \(z\) elements of a sequence \(X\). Additionally, let \(X \concat Y\) be the concatenation of two sequences \(X\) and \(Y\).

\paragraph{The optimal substructure.} Suppose \(T\) is a solution; i.e. it is a non-decreasing subsequence of \(S\) of maximal length. Now suppose the last value of \(T'_m\) is \(s_a\).

We can be certain that \(T'_m\) is the longest non-decreasing subsequence of \(S'_a\) (i.e. the first \(a\) terms of \(S\)) such that the last element of the subsequence is less than or equal to \(t_{m+1}\).

\paragraph{The proof.} Suppose \(T'_m\) is not the longest subsequence within the conditions specified earlier. Meaning there exists \(U\) which is a longer subsequence of \(S'_a\). Then we would be able to substitute the prefix \(T'_m\) for the \(U\) in the subsequence \(T\), and that would give a longer subsequence of \(S\), contradicting the maximality of \(T\).

\paragraph{The formula.} Let \(T\) denote the non-decreasing subsequence of \(S\) of maximal length. Additionally, let \(Q_i\) denote the maximal subsequence of \(S'_i\) which ends in \(s_i\).
\begin{align*}
	T   & = \displaystyle\max_{1 \leq i \leq n} \left( Q_i \right) \\
	Q_i & =
	\begin{cases}
		()                                                                            & i = 1 \\
		\displaystyle\max_{j < i \ : \ s_j \leq s_i} \left( Q_j \right) \concat (s_i) & i > 1
	\end{cases}
\end{align*}

In simpler terms, our solution \(T\) is obtained by first calculating a maximal subsequence which ends in each term \(s_i\) of \(S\), then taking the longest of those.

And in order to find a maximal subsequence which ends in \(s_i\), we first find a maximal subsequence within the terms before \(s_i\), which ends in a value smaller than or equal to \(s_i\), so that \(s_i\) may be concatenated to it.

\paragraph{The algorithm.} The iterative algorithm \textsc{LNDS} calculates the longest non-decreasing subsequence of the sequence \(S\). In this algorithm, \(p_i\) is set to the index of the predecessor of \(s_i\) in the sequence \(Q_i\). (If the predecessor is set to null, that indicates that there is no predecessor.) Further, we let \(l_i\) denote the length of \(Q_i\). Throughout the algorithm, each index \(i\) is inserted into a min-heap \(H_{l_i}\) whose key is \(s_i\). We use min-heaps so we can calculate \(\min(H_k)\) in constant time. The result, \(T\), is a stack whose top element is the beginning of the subsequence and whose last is the end of the subsequence.

\begin{algorithm}
	\begin{algorithmic}
		\Function{LNDS}{$S$}
		\If{\(|S| = 0\)}
		\Return ()
		\EndIf
		\State \(p_1 := \text{null}\) \Comment{The first element has no predecessor}
		\State \(l_1 := 1\) \Comment{The length of \(Q_1\) is 1}
		\State \(H_1 \leftarrow (s_1, 1)\) \Comment{Insert 1 into the min-heap of subsequences of length 1}
		\State \(h := 1\) \Comment{\(h\) is the length of the longest \(Q_i\)}
		\For{\(i\) from 2 to \(n\)}
		\For{\(k\) from \(h\) to 1}
		\State \(j := \min(H_k)\) \Comment{\(j\) is the last index of a \(Q_j\) s.t. \(j < i\) and \(s_j < s_i\)}
		\If{\(s_j \leq s_i\)} break \Comment{We found the correct \(j\)}
		\EndIf
		\EndFor
		\State \(p_i := j\)
		\State \(l_i := l_j + 1\)
		\State \(H_{l_i} \leftarrow (s_i, i)\)
		\State \(h := \max(h, l_i)\) \Comment{Update largest length}
		\EndFor
		\State \(j := \min(H_h)\) \Comment{The end of the longest subsequence}
		\While{\(j \neq \text{null}\)}
		\State \(T \leftarrow s_j\)
		\State \(j := p_j\)
		\EndWhile
		\State \Return \(T\)
		\EndFunction
	\end{algorithmic}
\end{algorithm}

\paragraph{Complexity analysis.} Throughout the course of the algorithm, we iterate over the sequence once, which is \(O(n)\) operations. For each of these iterations, we iterate over some of our \(h\) different min-heaps, which in the worst case is at most \(O(n)\). (On the average case it is a lot less, and I conjecture that even in the worst case it is in also less, since if \(h\) is very large, it means we often broke early on in the loop.) Additionally for each element in the sequence, we also perform a heap insertion, which is \(O(\lg(n))\).

Finally to obtain the final result, we select the elements in the largest subsequence, which is \(h\) operations, and \(h\) is \(O(n)\).

The complexity of the algorithm is therefore at most \(O(n^2)\). (And according to my conjecture, it would be closer to \(O(n\lg(n))\))

Regarding the space complexity, we keep two arrays \(p\) and \(l\), each of length \(n\), as well as \(h\) heaps, whose combined size is \(n\) (since each \(i\) is only inserted into one heap). Thus the algorithm uses \(O(n)\) extra space.

\section{Activity Selection Problem}

We are given a set of activities \(A = \{a_1, \dots, a_n\}\), and a single resource. Each activity \(a_i\) has a start time \(s_i\) and a finish time \(f_i\) such that \(f_i > s_i\). The time range of the activity \(a_i\) is the half-open interval \([s_i, f_i)\). Also given the start time \(S\) and the finish time \(F\) of the resource. (If for activity \(i\), \(s_i < S\) or \(f_i > F\) then this activity cannot use the resource.)

The goal is to find a subset \(A' \subseteq A\) of mutually compatible activities such that the resource is maximally used, i.e. such that the value of \(\sum_{a_i \in A'}(f_i - s_i)\) is maximal.

\subsection*{Part A}

In the first lecture of the course a similar problem was presented along with a greedy algorithm which could solve it. In that problem the goal was to select as many compatible activities as possible, whereas in this problem we aim for maximal use of the resource.

\subsection*{Part B}

To show that the greedy algorithm does not always return the correct result we will propose a counter-example. Suppose \(S = 0\) and \(F = 10\). Suppose the start times and finish times of the activities are as shown in table \ref{q3b-1}.

\begin{table}[hbtp]
	\centering
	\begin{tabular}{|c|ccc|}
		\hline
		\(i\)   & 1 & 2 & 3 \\
		\hline
		\(s_i\) & 0 & 5 & 0 \\
		\(f_i\) & 1 & 6 & 10 \\
		\hline
	\end{tabular}
	\caption{Start times and finish times of activities}
	\label{q3b-1}
\end{table}

The greedy algorithm would first select the activity which finishes first, namely \(a_1\), then would eliminate \(a_3\) which is incompatible with \(a_1\); then it would select \(a_2\), which is the activity with the second smallest finish time. So it would return \(\{a_1, a_2\}\), which uses the resource for \((1-0)+(6-5)=2\) units of time. However there is a solution which uses the resource for longer, namely \(\{a_3\}\) which uses it for 10 time units.

\subsection*{Part C}



\end{document}