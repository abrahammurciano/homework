\documentclass{article}

\usepackage{amsmath}
\usepackage{amssymb}
\usepackage{algorithm}
\usepackage[noend]{algpseudocode}		% for algorithms in pseudo code. Usage: \begin{algorithmic}
\MakeRobust{\Call}
\newcommand{\lang}{\mathcal{L}}

\setlength{\parskip}{\smallskipamount}

\title{Analysis of Algorithms \\
\medskip
\large Homework 6 -- P vs NP}
\author{Abraham Murciano \& Elad Harizy}

\begin{document}

\maketitle

\section*{Question 1}

\subsection*{Part A}

We are to prove that if the languages \(\lang_1\) and \(\lang_2\) are in P, meaning that there automata that can tell us whether a word is in the language or not in polynomial time (\(O(n*k)\) for some constant \(k\)), then \(\lang_1 \cup \lang_2 \in P\).

Since \(\lang_1\) and \(\lang_2\) can be decided in polynomial time, their union can also, as explained in Part B.

\subsection*{Part B}

We are told that the languages \(\lang_1, \lang_2\) can be decided in polynomial time using algorithms \(A_1, A_2\) respectively, with running times \(O(n^{k_1}), O(n^{k_2})\). To decide their union, one would have to decide each one individually, and decide their union based on their logical disjunction. Thus the complexity of deciding their union is \(O(n^{k_1} + n^{k_2})\), or \(O(n^{\max(k_1, k_2)})\), which is polynomial.

\section*{Question 2}

\subsection*{Part A}

We must prove that if \(\lang \in P\) then \(\forall k \in \mathbb{N}, \lang^k \in P\). Meaning that for any constant \(k\), we can decide the concatenation of the language to itself \(k\) times, in polynomial time.

We will use a lemma which states that if \(\lang_1, \lang_2 \in P\), then \(\lang_1\lang_2 \in P\). (Proof omitted.)

We will prove this by induction.

For \(k = 0, \lang^k = \lang^0 = \{\varepsilon\} \in P\).

Assume that for \(k = n, \lang^k = \lang^n \in P\).

Then for \(k = n + 1, \lang^k = \lang^{n+1} = \lang^n \lang\). However we know that both \(\land\) and \(\lang^n\) are in \(P\), so using our lemma, their concatenation, \(\lang^{n+1}\) must be in \(P\).

\subsection*{Part B}

Given that algorithm \(A_1\) decides \(\lang\) in \(O(n^c)\) time, we are to find the complexity of an algorithm \(A_2\) which decides \(\lang^k\) for some constant \(k\).

To decide \(\lang^2\), the complexity would be \(O((n^c)^2)\), or \(O(n^{2c})\). This is because after each character, we must check if the remainder of the input is also in \(\lang\). So if we repeat this process \(k\) times, the algorithm results in a complexity of \(O(n^{kc})\).

\end{document}