\documentclass[fleqn]{article}
\usepackage{preamble}

\title{Automata \& Formal Languages \\
\medskip
\large Homework 4 -- Non-Deterministic Finite Automata}
\author{Abraham Murciano}

\begin{document}

\maketitle

\begin{answers}

	\item % 1
		\begin{enumerate}
			\item % a
				Figure \ref{q1a} shows an NFA without \(\varepsilon\) transitions or multiple start states which identifies the language \(\{\varepsilon\}\).
				\begin{figure}[htb]
					\centering
					\begin{statediagram}
						\node[state, start, accepted] {\(q_0\)};
						\node[state, right=of start] (q1) {\(q_1\)};
						\draw[input] (start) to["{\(\Sigma\)}"] (q1);
					\end{statediagram}
					\caption{NFA for question 1a}
					\label{q1a}
				\end{figure}

			\item % b
				Figure \ref{q1b} shows an NFA which accepts the language of words over \{a,b\} that end in ``abb''.
				\begin{figure}[htb]
					\centering
					\begin{statediagram}
						\node[state, start] {\(q_0\)};
						\node[state, right=of start] (a) {\(q_1\)};
						\node[state, right=of a] (b1) {\(q_2\)};
						\node[state, accepted, right=of b1] (b2) {\(q_3\)};

						\draw[input, loop above] (start) to["{\(\Sigma\)}", above] (start);
						\draw[input] (start) to["{a}"] (a);
						\draw[input] (a) to["{b}"] (b1);
						\draw[input] (b1) to["{b}"] (b2);
					\end{statediagram}
					\caption{NFA for question 1b}
					\label{q1b}
				\end{figure}

			\item % c
				Figure \ref{q1c} shows an NFA which accepts the language of words over \(\Sigma = \{\text{a}, \text{b}\}\) that contain ``aa'' or that have an odd number of ``b''s.
				\begin{figure}[htb]
					\centering
					\begin{statediagram}
						\node[state, start] {\(q_0\)};
						\node[state, above right=of start] (a1) {\(q_1\)};
						\node[state, accepted, right=of a1] (a2) {\(q_2\)};
						\node[state, accepted, below right=of start] (b_odd) {\(q_3\)};

						\draw[input] (start) to["{a}", sloped] (a1);
						\draw[input, bend left=40] (start) to["{b}"] (b_odd);
						\draw[input, loop above] (start) to["{a}"] (start);
						\draw[input] (a1) to["{a}", sloped] (a2);
						\draw[input, loop right] (a2) to["{\(\Sigma\)}"] (a2);
						\draw[input, bend left=40] (b_odd) to["{b}"] (start);
						\draw[input, bend right=40] (b_odd) to["{a}" right] (a1);
						\draw[input, loop right] (b_odd) to["{a}"] (b_odd);
					\end{statediagram}
					\caption{NFA for question 1c}
					\label{q1c}
				\end{figure}
		\end{enumerate}
	\item % 2
		Figure \ref{q2-f} shows a conversion of the NFA in figure \ref{q1c} into a DFA. Table \ref{q2-t} shows a table which we can use to aid us in the construction of the DFA.
		\begin{table}
			\centering
			\begin{tabular}{||C|C|C|C||}
				\hline
				\text{New label} & \text{Current state} & \text{Transition on a} & \text{Transition on b} \\
				\hline
				r_0              & \{q_0\}              & \{q_0, q_1\}           & \{q_3\} \\
				r_1              & \{q_0, q_1\}         & \{q_0, q_1, q_2\}      & \{q_3\} \\
				r_2              & \{q_3\}              & \{q_1, q_3\}           & \{q_0\} \\
				r_3              & \{q_0, q_1, q_2\}    & \{q_0, q_1, q_2\}      & \{q_2, q_3\} \\
				r_4              & \{q_1, q_3\}         & \{q_1, q_2, q_3\}      & \{q_0\} \\
				r_5              & \{q_2, q_3\}         & \{q_1, q_2, q_3\}      & \{q_0, q_2\} \\
				r_6              & \{q_1, q_2, q_3\}    & \{q_1, q_2, q_3\}      & \{q_0, q_2\} \\
				r_7              & \{q_0, q_2\}         & \{q_0, q_1, q_2\}      & \{q_2, q_3\} \\
				\hline
			\end{tabular}
			\caption{A table to translate the NFA in figure \ref{q1c} to the DFA in figure \ref{q2-f}}
			\label{q2-t}
		\end{table}
		\begin{figure}[htb]
			\centering
			\begin{statediagram}
				\node[state, start] {\(r_0\)};
				\node[state, above right=of start] (r1) {\(r_1\)};
				\node[state, accepted, below right=of start] (r2) {\(r_2\)};
				\node[state, accepted, right=of r1] (r3) {\(r_3\)};
				\node[state, accepted, right=of r2] (r4) {\(r_4\)};
				\node[state, accepted, right=of r3] (r5) {\(r_5\)};
				\node[state, accepted, right=of r4] (r6) {\(r_6\)};
				\node[state, accepted, below right=of r5] (r7) {\(r_7\)};

				\draw[input] (start) to["{a}", sloped] (r1);
				\draw[input] (start) to["{b}", sloped] (r2);

				\draw[input] (r1) to["{a}"] (r3);
				\draw[input] (r1) to["{b}"] (r2);

				\draw[input] (r2) to["{a}"] (r4);
				\draw[input, bend left=50] (r2) to["{b}", sloped] (start);

				\draw[input, loop above] (r3) to["{a}"] (r3);
				\draw[input] (r3) to["{b}"] (r5);

				\draw[input] (r4) to["{a}"] (r6);
				\draw[input, bend left=75] (r4) to["{b}" below, sloped] (start);

				\draw[input] (r5) to["{a}"] (r6);
				\draw[input] (r5) to["{b}" below, sloped] (r7);

				\draw[input, loop below] (r6) to["{a}"] (r6);
				\draw[input] (r6) to["{b}", sloped] (r7);

				\draw[input, bend right=75] (r7) to["{a}" above, sloped] (r3);
				\draw[input, bend right=40] (r7) to["{b}" below, sloped] (r5);
			\end{statediagram}
			\caption{A DFA equivalent to the automata in figure \ref{q1c}}
			\label{q2-f}
		\end{figure}

\end{answers}

\end{document}
